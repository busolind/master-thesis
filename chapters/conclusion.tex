\phantomsection
\addcontentsline{toc}{chapter}{Conclusion}  
\chapter*{Conclusion}
\pagestyle{plain}
The work presented in this thesis has conducted a performance evaluation of a simulated ToD service over a 5G network in a multitude of different scenarios.

An overview and definition of a Tele-Operated Driving service were provided, as described by the 5G Automotive Association, together with its technical and functional requirements and a discussion of some possible use cases.

In order to conduct the performance evaluation, a software basis made up of CARLA driving simulator and OMNeT++ network simulator, mated through a message-oriented middleware that enables co-simulation, was employed. It was expanded to allow for the testing of specific scenarios involving more than one vehicle and special spawning conditions for obstacles.
More specifically, a long route spanning highway and urban sections of the chosen map was evaluated
using increasing levels of network background generated in two different ways. A couple of scenarios involving obstacles on the vehicle's path were also devised, both with the obstacle in a fixed position from the start in an urban environment and suddenly appearing in a city and highway context.
Each simulated scenario was evaluated over a set of parameter configurations that condition the bandwidth needed by the Host Vehicle and the amount of background network traffic.

An analysis of the recorded simulation data was provided for relevant parameter configurations of each considered scenario, and it was shown that Tele-Operated Driving is certainly a possibility under the right conditions and with the correct setup. It is possible to achieve safe operation within the specifications and requirements of the 5GAA even with a basic 5G network configuration.

The most impactful parameter in the success or failure of this type of service is the overall latency, or RTT, which was generally lower with a numerology index $\mu$ of 1 and influenced by channel quality and network load.

Future work might include some experimentation with Carrier Aggregation for the networking side and tests in situations with external vehicle traffic using different agents as ToD operators.

Achieving real-time operation of the simulation would allow for testing with human operators and verifying the effect the network delay has on their ability to keep control of the vehicle.

An interesting expansion of the simulation framework would be towards being able to gather additional data from the infrastructure itself and leveraging multi-access edge computing (MEC) \cite{mec_collision_avoidance}, where collisions can be more easily avoided by assessing and predicting movements made by other road users.