\phantomsection
\addcontentsline{toc}{chapter}{Introduction}  
\chapter*{Introduction}

Self-driving vehicles have been the topic of extensive research and experimentation for a number of years, with auto makers and tech companies announcing major breakthroughs and the viability of the technology for public use getting closer and closer. There have been some public tests and recent service deployments, like the San Francisco robo-taxi service.

Nevertheless, even as constant training for AI models makes them better by the day and advancements in sensor technology make them more reliable, there will be situations where human intervention is required, be it a software crash, a sensor malfunction, or simply an edge case that is not contemplated by the training the model has received. Until recently, this has been accomplished by simply having a human in the vehicle at all times ready to intervene, but a possible novel solution is called \textit{Tele-Operated Driving}.
Tele-Operated Driving (ToD) is a service that allows an operator, human or machine, to remotely take control of a Host Vehicle (HV) and complete the required driving tasks. This type of service is enabled by advancements in cellular network technology such as 5G.
In the case of a human operator, a virtual cockpit is set up to reproduce the vehicle's environment as it is actively captured by on-board sensors and cameras, which are also used for the autonomous driving tasks. The operator then sends back control inputs that are actuated by the vehicle.

Studies were performed using previous technological iterations, i.e.\ 4G LTE. In \cite{remote_driving_lte_network}, \textit{Liu et al.} conducted experiments on remote driving over an LTE network, collecting data from both a scaled-size prototype and real-world measurements of publicly accessible mobile networks. Using a low-definition video feed, they registered RTT values between 56ms and 358ms, averaging around 100ms but with a lot of variance. In their experiments with human candidates, they concluded that with current technology at the time, remote driving was still a questionable endeavor, mainly due to the variability of network conditions.

Previous work \cite{valeriopaper} has laid the groundwork for realistic simulation of both vehicle physics and a 5G communication network by creating a middleware that interfaces OMNeT++, INET, and the Simu5G frameworks with the CARLA driving simulator. The goal of this thesis is to carry out a performance evaluation of the ToD infrastructure against a set of scenarios, that take the form of a route that the HV must navigate which includes both urban and highway sections and some experiments with obstacles that force the vehicle to perform an emergency stop. Each of those scenarios is tested under a set of parameter configurations that influence the need for resources of the Host Vehicle and the amount of background other users generate, putting an additional load on the network.

The ToD simulator framework needed expanding in order to allow for the appropriate scenarios to be simulated as part of this thesis work. To accomplish this, features were added that allow an obstacle to be placed on the map either since the start of the simulation or only when the HV is within a certain distance from its spawn point. A novel background generation model was included as well, in the form of \textit{background vehicles}, one or more, that drive around the world in the driving simulator much like the HV, but are linked to background generators in the network simulator that then effectively move around, differently from the default method of users in fixed positions around each base station.

The simulation results show that Tele-Operated driving over 5G is feasible but heavily influenced by the radio conditions, most importantly the network delay which is affected by factors including the required bandwidth, the channel quality, the volume of other traffic on the shared medium, and the handovers between different base stations.

This thesis is organized as follows:
Chapter~\ref{chap:tele_operated_driving} explains the concept of Tele-Operated Driving in detail as defined by sector associations, with its characteristics, variations, and proposed requirements. Chapter~\ref{chap:tod_simulator} describes the software foundation used to perform the simulations required for the performance evaluation, starting from an overview of the two simulators used for the network and driving side and continuing with the middleware that ties them together. Chapter~\ref{chap:simulated_scenarios} details the simulated scenarios including the chosen map from the driving simulator and parameter configurations. Finally, Chapter~\ref{chap:results} exposes the results of the analysis performed on the simulation data, together with the description of the evaluated metrics.
